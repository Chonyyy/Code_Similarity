\chapter*{Introducción}\label{chapter:introduction}
\addcontentsline{toc}{chapter}{Introducción}

El análisis de similitud de código es un área de Ciencias de la Computación que se centra en la comparación de fragmentos de código para identificar similitudes y diferencias. Este enfoque es fundamental en la detección de plagio y en la revisión de código, ya que proporciona métricas objetivas para la evaluación de la calidad del código y la adopción de buenas prácticas de programación. 

\section*{Motivación}

En el contexto académico, la detección de similitud de código y el análisis de plagio representan un desafío en la evaluación de proyectos de programación. Con el crecimiento constante de los recursos de ciencias de la computación, los educadores enfrentan la tarea de garantizar la originalidad en las soluciones presentadas por los estudiantes. Este problema no solo afecta la integridad académica, sino que también dificulta evaluar de manera justa el aprendizaje y la comprensión de conceptos.  

El plagio de código puede adoptar diversas formas, desde la copia literal hasta modificaciones superficiales como el cambio de nombres de variables, la reorganización de bloques de código o la reescritura con constructos equivalentes. Estas transformaciones suelen evadir los enfoques tradicionales de detección basados en comparaciones textuales o métricas básicas de similitud. Como resultado, la detección efectiva requiere sistemas que puedan identificar similitudes semánticas y estructurales, independientemente de las transformaciones realizadas.  

Esta investigación surge de la necesidad de dotar a las instituciones educativas de herramientas avanzadas que puedan analizar código con precisión. Al combinar técnicas de análisis estático, modelos de representación estructural y métodos de aprendizaje automático, esta tesis propone un marco robusto para la detección de similitud de código. Este enfoque no solo busca resolver los problemas prácticos del plagio en entornos académicos, sino también contribuir al entendimiento teórico del problema y ofrecer una base para futuras investigaciones en el área.  


\section*{Objetivos Generales} 

\renewcommand{\labelenumi}{\Roman{enumi}.}
\begin{enumerate}
	\item Desarrollar un marco de análisis de similitud de código que combine la extracción de características mediante árboles de sintaxis abstracta (AST) con técnicas avanzadas de aprendizaje automático para mejorar la precisión y eficiencia en la detección de similitudes.
	\item Contribuir al conocimiento académico y práctico en el campo del análisis de código, proporcionando herramientas y metodologías que puedan ser utilizadas tanto en entornos educativos como profesionales.
\end{enumerate}
 
\section*{Objetivos Específicos}

\renewcommand{\labelenumi}{\Roman{enumi}.}
\begin{enumerate}
	\item {\bf Extracción y Procesamiento de AST para la Comparación de Código}\\
El primer objetivo es diseñar e implementar un sistema que utilice Árboles de Sintaxis Abstracta (AST) para analizar y comparar proyectos de código. En la primera etapa, se utiliza la herramienta ANTLR y se modifica su gramática para C\# con el fin de generar los AST de los proyectos, a partir de los cuales se extraen los \textit{features} relevantes para su posterior análisis.En la segunda etapa, los conjuntos de \textit{features} extraídos de cada proyecto se convierten en vectores de características, que servirán como base para las comparaciones entre códigos. Este enfoque busca capturar similitudes estructurales y semánticas en los proyectos.  
	
	\item {\bf Integrar Técnicas de Aprendizaje Automático para la Detección de Similitudes} \\
El segundo objetivo es integrar técnicas de aprendizaje automático para mejorar la detección de similitudes en el código. Esto implica entrenar un modelo de aprendizaje supervisado utilizando un conjunto de datos conformado por proyectos de C\#. Estos modelos deberán capturar patrones complejos y relaciones estructurales en el código, proporcionando una visión precisa de las similitudes. La implementación de estas técnicas permitirá comparar su desempeño con los métodos tradicionales, evaluando mejoras en precisión y eficiencia.

	\item {\bf Desarrollar y Validar una Herramienta Práctica} \\
El tercer objetivo es desarrollar una herramienta de software que implemente las técnicas avanzadas desarrolladas, facilitando su uso por desarrolladores y educadores. Esta herramienta deberá ser práctica y accesible, permitiendo su integración en entornos de desarrollo reales. La validación de la herramienta se llevará a cabo en escenarios prácticos, como la detección de plagio en tareas de programación. Esto no solo demostrará la efectividad de las técnicas implementadas, sino que también garantizará que la herramienta sea útil y relevante para los usuarios finales.


\end{enumerate}

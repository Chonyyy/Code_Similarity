\chapter*{Introducción}\label{chapter:introduction}
\addcontentsline{toc}{chapter}{Introducción}

El análisis de similitud de código es un campo de gran escala en Ciencias de la Computación, especialmente en áreas como la detección de plagio, la revisión de código y la asistencia en programación. Este análisis permite comparar fragmentos de código para identificar similitudes y diferencias, proporcionando información valiosa para la refactorización, la mejora de la calidad del código y la promoción de buenas prácticas de programación. \\

\section{Motivación}
La creciente complejidad y volumen del software moderno ha generado la necesidad de herramientas más sofisticadas para analizar y comprender el código. La similitud de código es una métrica importante que puede ayudar a detectar duplicaciones, plagio y patrones reutilizables, facilitando la mejora continua del software. Además, con el avance de las técnicas de aprendizaje automático y el procesamiento del lenguaje natural, se han abierto nuevas posibilidades para realizar este análisis de manera más efectiva y precisa.

\section{Problemática}
A pesar de los avances en el análisis de similitud de código, aún existen desafíos significativos. Las técnicas tradicionales, basadas en comparaciones textuales simples, son limitadas en su capacidad para capturar la estructura y semántica del código. Además, los enfoques más avanzados, como los basados en árboles de sintaxis abstracta (AST) y modelos de aprendizaje profundo, pueden ser computacionalmente costosos y difíciles de implementar a gran escala. La precisión y la eficiencia siguen siendo áreas críticas de mejora, especialmente en contextos donde el código es altamente variable y complejo.

\section{Objetivos Generales} 

\renewcommand{\labelenumi}{\Roman{enumi}.}
\begin{enumerate}
	\item Desarrollar un marco de análisis de similitud de código que combine la extracción de características mediante árboles de sintaxis abstracta (AST) con técnicas avanzadas de aprendizaje automático, con el fin de mejorar la precisión y eficiencia en la detección de similitudes.
	\item Contribuir al conocimiento académico y práctico en el campo del análisis de código, proporcionando herramientas y metodologías que puedan ser utilizadas tanto en entornos educativos como profesionales.
\end{enumerate}
 
\section{Objetivos Específicos}

\renewcommand{\labelenumi}{\Roman{enumi}.}
\begin{enumerate}
	\item {\bf Implementar y Evaluar Algoritmos de Comparación Basados en AST}\\
	El primer objetivo es diseñar e implementar algoritmos de comparación que utilicen Árboles de Sintaxis Abstracta (AST) para identificar similitudes en el código. Este proceso implica desarrollar un método detallado para extraer subárboles de los AST y compararlos, capturando similitudes estructurales en diferentes niveles de granularidad. La efectividad de estos algoritmos se evaluará en términos de precisión y tiempo de procesamiento, comparándolos con los métodos tradicionales de comparación de código. Esto permitirá determinar si los enfoques basados en AST ofrecen ventajas significativas en el análisis de similitud de código.
	
	\item {\bf Integrar Técnicas de Aprendizaje Automático para la Detección de Similitudes} \\
El segundo objetivo es integrar técnicas de aprendizaje automático para mejorar la detección de similitudes en el código. Esto implica entrenar modelos de aprendizaje supervisado y no supervisado utilizando un conjunto de datos conformado por proyectos de C\# de la facultad. Estos modelos deberán capturar patrones complejos y relaciones estructurales en el código, proporcionando una visión más profunda y precisa de las similitudes. La implementación de estas técnicas permitirá comparar su desempeño con los métodos tradicionales, evaluando mejoras en precisión y eficiencia.

	\item {\bf Desarrollar y Validar una Herramienta Práctica} \\
El tercer objetivo es desarrollar una herramienta de software que implemente las técnicas avanzadas desarrolladas, facilitando su uso por desarrolladores y educadores. Esta herramienta deberá ser práctica y accesible, permitiendo su integración en entornos de desarrollo reales. La validación de la herramienta se llevará a cabo en escenarios prácticos, como la detección de plagio en tareas de programación. Esto no solo demostrará la efectividad de las técnicas implementadas, sino que también garantizará que la herramienta sea útil y relevante para los usuarios finales.


\end{enumerate}

\chapter{Propuesta de solución}\label{chapter:proposal}

La implementacion  de este trabajo esta dividida en dos partes fundamentales: Extraccion del AST y trabajo con machine learning.

- Extraccion del AST: En esta parte se utiliza la herramienta ANTLR de C\# para crear el ast de los proyectos. Luego de este ast se extraen los features para su posterior uso.

- Trabajo con machine learning: Se estudian diferentes modelos para determinar la semejanza entre 2 proyectos.


\section{Extraccion del AST}

ANTLR (Another Tool for Language Recognition) es una poderosa herramienta que se utiliza para generar analizadores léxicos y sintácticos a partir de gramáticas definidas por el usuario. Es ampliamente utilizada en la compilación y el procesamiento de lenguajes, ya que permite transformar el código fuente en estructuras de datos que pueden ser fácilmente manipuladas. En este caso, ANTLR se emplea para extraer Árboles de Sintaxis Abstracta (AST) a partir del código fuente de programas escritos en C\#. Fue necesario hacer ligeras modificaciones sobre la gramática, pues ANTLR tenia una gramatica desactualizada. \\

Una vez que se ha generado el AST, se puede manipular y analizar utilizando las estructuras de datos proporcionadas por ANTLR. Por ejemplo, se pueden recorrer los nodos del árbol, extraer información específica o transformar el AST para diferentes propósitos, en este caso se utilizó el listener proporcionado por ANTLR para recorrer el árbol y extraer los featurues. \\

\section{Extraccion de features}

En el análisis de similitud de código y detección de patrones, es crucial extraer características relevantes que capturen la estructura y el comportamiento del código. Para este propósito, se implementó una clase denominada {\bf FeatureExtractorListener}, que extiende la funcionalidad de ANTLR para analizar el código fuente en C\#. A continuación, se presenta una descripción detallada del proceso de extracción de características y la importancia de cada característica extraída. \\

\begin{enumerate}
	\item Estructura del AST:
    		\begin{itemize}
			\item {\bf total\_nodes:} Número total de nodos en el AST.
			\item {\bf max\_depth:} Profundidad máxima del AST.
		\end{itemize}
		
	 \item Declaraciones y Variables:
    \begin{itemize}
        \item {\bf variables:} Número de variables locales.
        \item {\bf constants:} Número de constantes declaradas.
        \item {\bf variable\_names:} Conjunto de nombres de variables y sus tipos.
        \item {\bf number\_of\_tuples:} Número de variables de tipo tupla.
        \item {\bf lists:} Número de listas declaradas.
        \item {\bf dicts:} Número de diccionarios declarados.
    \end{itemize}
    Las variables y las constantes son fundamentales para entender el estado y el flujo de datos en el código. La variedad y el tipo de estructuras de datos utilizadas (tuplas, listas, diccionarios) también proporcionan información sobre el estilo de programación y la complejidad del código.

    \item Declaraciones de Métodos y Clases:
    \begin{itemize}
        \item {\bf methods:} Número de métodos declarados.
        \item {\bf method\_names:} Conjunto de nombres de métodos.
        \item {\bf method\_return\_types:} Conjunto de tipos de retorno de métodos.
        \item {\bf method\_parameters:} Lista de parámetros de métodos.
        \item {\bf classes:} Número de clases declaradas.
        \item {\bf class\_names:} Conjunto de nombres de clases.
        \item {\bf abstract\_classes:} Número de clases abstractas.
        \item {\bf sealed\_classes:} Número de clases selladas.
        \item {\bf interfaces:} Número de interfaces declaradas.
        \item {\bf interface\_names:} Conjunto de nombres de interfaces.
    \end{itemize}
    La estructura y los nombres de los métodos y clases proporcionan información sobre la organización y modularidad del código. Los métodos y sus parámetros son esenciales para entender la funcionalidad del código, mientras que las clases y sus tipos (abstractas, selladas) indican la arquitectura de la aplicación.

    \item Estructuras de Control:
    \begin{itemize}
        \item {\bf control\_structures\_if:} Número de sentencias if.
        \item {\bf control\_structures\_switch:} Número de sentencias switch.
        \item {\bf control\_structures\_for:} Número de bucles for.
        \item {\bf control\_structures\_while:} Número de bucles while.
        \item {\bf control\_structures\_dowhile:} Número de bucles do-while.
        \item {\bf try\_catch\_blocks:} Número de bloques try-catch.
    \end{itemize}
    Las estructuras de control son fundamentales para comprender el flujo del programa y su lógica. Un mayor número de estructuras de control indica una lógica más compleja y ramificada.

    \item Modificadores y Accesibilidad:
    \begin{itemize}
        \item {\bf access\_modifiers\_public:} Número de elementos públicos.
        \item {\bf access\_modifiers\_private:} Número de elementos privados.
        \item {\bf access\_modifiers\_protected:} Número de elementos protegidos.
        \item {\bf access\_modifiers\_internal:} Número de elementos internos.
        \item {\bf access\_modifiers\_static:} Número de elementos estáticos.
        \item {\bf access\_modifiers\_protected\_internal:} Número de elementos protegidos internos.
        \item {\bf access\_modifiers\_private\_protected:} Número de elementos privados protegidos.
    \end{itemize}
    Los modificadores de acceso proporcionan información sobre la encapsulación y visibilidad de los componentes del código. La prevalencia de ciertos modificadores puede indicar prácticas de diseño y seguridad en el código.

    \item Modificadores Específicos:
    \begin{itemize}
        \item {\bf modifier\_readonly:} Número de elementos readonly.
        \item {\bf modifier\_volatile:} Número de elementos volatile.
        \item {\bf modifier\_virtual:} Número de elementos virtual.
        \item {\bf modifier\_override:} Número de elementos override.
        \item {\bf modifier\_new:} Número de elementos new.
        \item {\bf modifier\_partial:} Número de elementos partial.
        \item {\bf modifier\_extern:} Número de elementos extern.
        \item {\bf modifier\_unsafe:} Número de elementos unsafe.
        \item {\bf modifier\_async:} Número de elementos async.
    \end{itemize}
    Estos modificadores específicos indican características avanzadas y patrones de diseño en el código, como la concurrencia (async), la seguridad (unsafe) y la herencia (override, virtual).

    \item Llamadas a Librerías y LINQ:
    \begin{itemize}
        \item {\bf library\_call\_console:} Número de llamadas a la librería Console.
        \item {\bf library\_call\_math:} Número de llamadas a la librería Math.
        \item {\bf linq\_queries\_select:} Número de consultas LINQ Select.
        \item {\bf linq\_queries\_where:} Número de consultas LINQ Where.
        \item {\bf linq\_queries\_orderBy:} Número de consultas LINQ OrderBy.
        \item {\bf linq\_queries\_groupBy:} Número de consultas LINQ GroupBy.
        \item {\bf linq\_queries\_join:} Número de consultas LINQ Join.
        \item {\bf linq\_queries\_sum:} Número de consultas LINQ Sum.
        \item {\bf linq\_queries\_count:} Número de consultas LINQ Count.
    \end{itemize}
    Las llamadas a librerías y consultas LINQ proporcionan información sobre el uso de funcionalidades estándar y el manejo de colecciones de datos en el código.

    \item Otras Características:
    \begin{itemize}
        \item {\bf number\_of\_lambdas:} Número de expresiones lambda.
        \item {\bf number\_of\_getters:} Número de métodos get.
        \item {\bf number\_of\_setters:} Número de métodos set.
        \item {\bf number\_of\_namespaces:} Número de espacios de nombres.
        \item {\bf enums:} Número de enumeraciones.
        \item {\bf enum\_names:} Conjunto de nombres de enumeraciones.
        \item {\bf delegates:} Número de delegados.
        \item {\bf delegate\_names:} Conjunto de nombres de delegados.
        \item {\bf node\_count:} Conteo de nodos por tipo.
    \end{itemize}
    Estas características adicionales proporcionan una visión más completa de las capacidades del código, su organización y las prácticas de programación utilizadas.

 
\end{enumerate}
    
        
La extracción de características con FeatureExtractorListener permite capturar una amplia gama de aspectos del código fuente en C\#, desde su estructura y complejidad hasta los patrones de diseño y las prácticas de programación. Esta información es crucial para tareas como la detección de similitudes de código, la evaluación de la calidad del código y la identificación de posibles plagios. La implementación y el análisis detallado de estas características proporcionan una base sólida para mejorar la precisión y efectividad de las herramientas de análisis de código.

\section{Preparacion del dataset}

Para preparar el dataset utilizado en el análisis de similitud de código, se requirió convertir los nombres de variables, métodos y otros identificadores en vectores de características numéricas. Este proceso se llevó a cabo utilizando la técnica de embeddings con Word2Vec. A continuación, se explica qué es Word2Vec, cómo funciona y por qué se eligió para esta tarea \cite{mikolov2013efficient}. \\

\subsubsection{Uso de Word2Vec en la Preparación del Dataset}
Word2Vec es una técnica de aprendizaje profundo que transforma palabras en vectores de números de alta dimensión, conocidos como embeddings. Los modelos Word2Vec se entrenan utilizando grandes corpus de texto y capturan relaciones semánticas entre las palabras.  \\

En el contexto del análisis de similitud de código, los nombres de variables, métodos y otros identificadores en el código fuente juegan un papel crucial. Estos nombres pueden proporcionar información semántica valiosa sobre la funcionalidad y el propósito de diferentes partes del código. Por ejemplo, los identificadores que se usan de manera similar en diferentes contextos tendrán embeddings similares. Sin embargo, los identificadores en el código no están estructurados de manera que las máquinas puedan comprender fácilmente sus relaciones semánticas. Para abordar este desafío, se utilizó Word2Vec para convertir estos identificadores en embeddings. El proceso involucró los siguientes pasos:

\begin{enumerate}
	\item Extracción de Identificadores: Se extrajeron todos los nombres de variables, métodos, clases, interfaces, enumeraciones y delegados del código fuente utilizando la clase FeatureExtractorListener.
	
	\item Entrenamiento de Word2Vec: Se utilizó un corpus de identificadores extraídos de múltiples proyectos de C\# para entrenar el modelo Word2Vec. El modelo aprendió las relaciones semánticas entre los diferentes identificadores en el contexto del código.
	
	\item Conversión a Embeddings: Cada identificador extraído se convirtió en un vector de características numéricas utilizando el modelo Word2Vec entrenado. Luego se halla el promedio entre todos los vectores por feature correspondiente para asegurar que todos las caracteristicas de vectores tengan la misma dimensión. Estos vectores capturan la semántica y el contexto de los identificadores en el código.
	 
\end{enumerate}

\subsubsection{Nuevo dataset}
Para maximizar la cantidad de datos disponibles y reflejar de manera efectiva la similitud entre proyectos, se creó un nuevo dataset que contiene todos los pares posibles (2 a 2) de proyectos del conjunto de datos original. En este proceso, para cada par de proyectos, se calcula y almacena la diferencia entre sus vectores correspondientes.

Este enfoque tiene varias ventajas significativas:

\begin{itemize}
	\item {\bf Incremento en la Cantidad de Datos:} Generar todos los pares posibles de proyectos incrementa exponencialmente el número de instancias en el dataset, proporcionando una base de datos más rica y diversa para entrenar modelos de aprendizaje automático.

	\item {\bf Captura de Relaciones Detalladas:} Al almacenar la diferencia entre los vectores de cada par de proyectos, se capturan las distancias y relaciones específicas entre todos los proyectos. Esto permite que el modelo de machine learning pueda aprender las sutilezas de las similitudes y diferencias entre distintos proyectos.
	
	\item {\bf Mejora en la Precisión del Modelo:} Con un mayor volumen de datos y la inclusión de las distancias entre pares, se espera que el modelo tenga un mejor desempeño en la tarea de detección de similitudes. La precisión del modelo se ve beneficiada al disponer de más ejemplos que reflejan una amplia gama de variaciones y similitudes.
	
	\item {\bf Refinamiento de las Métricas de Similitud:} Este método permite que se utilicen métricas de similitud más refinadas y precisas, ya que cada par de proyectos se compara de manera detallada. 

\end{itemize}

En resumen, la creación de este nuevo dataset con todos los pares posibles y sus diferencias vectoriales no solo aumenta la cantidad de datos disponibles, sino que también enriquece la información sobre las relaciones entre proyectos, mejorando así la capacidad del modelo para detectar similitudes de manera precisa y eficiente.  

\section{Probando diferentes modelos}

\subsubsection{SNN}
La primera solución abordada fue implementar una red neuronal siamesa para comparar dos proyectos. Estas redes funcionan con dos entradas, procesando cada una a través de la misma arquitectura de red para obtener representaciones vectoriales comparables. La red neuronal siamesa mide la distancia entre estos vectores para determinar la similitud entre los proyectos.\\

Para esta solución, se separaron todos los posibles pares de proyectos y se les asignó una etiqueta: 0 si los proyectos no son copias (ambos originales) y 1 si los proyectos son copias. Los datos se convirtieron en vectores de características y el conjunto de datos se constituyó con los pares de vectores correspondientes y sus etiquetas.\\

Sin embargo, surgió el problema de la generación de copias de proyectos para etiquetar correctamente los datos. Esto llevó a la necesidad de buscar otra solución más efectiva a la carencia de datos etiquetados.


\subsubsection{Clustering}

El clustering es un proceso de aprendizaje no supervisado, utilizado para agrupar datos sin conocer de antemano la etiqueta o categoría de cada dato. En lugar de depender de etiquetas predefinidas, los algoritmos de clustering analizan la proximidad y similitud entre los datos para formar grupos o clústeres basados en características compartidas. Cada clúster agrupa datos que son similares entre sí, pero que difieren de los datos en otros clústeres. Este enfoque es particularmente útil en situaciones donde se quiere descubrir patrones, estructuras o relaciones inherentes sin la necesidad de etiquetas predefinidas.

Se aplicaron distintos enfoques de clustering, entre ellos:
\begin{itemize}
    \item \textbf{K-Means}: Agrupa datos en \(k\) clústeres basándose en la distancia euclidiana a los centroides, que se ajustan iterativamente. Aquí se hizo uso del método del codo para determinar el parámetro \(k\) a utilizar.
    \item \textbf{DBSCAN (Density-Based Spatial Clustering of Applications with Noise)}: Identifica clústeres basados en la densidad de puntos, permitiendo descubrir clústeres de forma arbitraria y detectar ruido.
    \item \textbf{Agglomerative Clustering}: Un enfoque jerárquico que une iterativamente los puntos o clústeres más cercanos, formando un dendrograma.
    \item \textbf{MeanShift}: Agrupa datos desplazando iterativamente los puntos hacia la densidad máxima de puntos más cercana, identificando modos en la densidad de datos.
\end{itemize}

Aunque algunos de estos algoritmos mostraron buenas métricas en ciertos aspectos, no lograron resolver completamente el problema planteado. 

\section{Modelos utilizados}

\subsubsection{One-Class SVM}

\subsubsection{Isolation forest}



\chapter{Detalles de Implementación y Experimentos}\label{chapter:implementation}

Para llegar a la solucion final se tuvo que pasar por una gran cantidad de pruebas. Primeramente se tuvo limitaciones en cuanto al dataset, no existen datos suficiente para crear un dataset de gran escala y tampoco se puedo generar automaticamente copias de proyectos, por lo cual se decidio aplicar clustering sobre los datos. \\

	Se aplican distintos enfoques de clustering, como K-Mean, DBSCAN, Agglomerative clustering, MeanShift para ambos datasets y se aplicandistintas metricas para comprobar su efectividad como: Silhouette, Davies-Bouldin y Calinski-Harabasz.
 
\begin{conclusions}

El análisis de similitud de código ha avanzado desde sus inicios, convirtiéndose en una herramienta para abordar problemas en los ámbitos académico. En esta tesis, se analizan los avances en representación estructural, como el uso de Árboles de Sintaxis Abstracta (AST), y el desarrollo de modelos basados en aprendizaje profundo que permiten detectar similitudes en código más allá de comparaciones textuales. Los AST, al capturar la estructura jerárquica del código, permiten realizar comparaciones basadas en lógica y semántica, evitando la dependencia de detalles superficiales. El proceso de actualización de la gramática de lenguajes como C\# garantiza la representación de nuevas características en los AST, manteniendo la aplicabilidad de las herramientas de análisis frente a la evolución de los lenguajes de programación.

Las redes neuronales siamesas permiten comparar representaciones vectoriales de fragmentos de código, incluso en casos con modificaciones superficiales. Estas redes, configuradas para procesar pares de entradas mediante ramas idénticas con pesos compartidos, miden similitudes enfocándose en las características más relevantes. Este enfoque permite detectar relaciones semánticas y estructurales que no pueden captarse mediante métodos tradicionales basados en características superficiales.

En conclusión, esta investigación evalúa las metodologías propuestas y establece un marco para el desarrollo de herramientas más precisas y adaptadas. Las contribuciones de esta tesis, que incluyen el uso de AST, el refinamiento de redes neuronales siamesas y la actualización de gramáticas, plantean oportunidades para investigaciones futuras en el análisis de similitud de código.

\end{conclusions}

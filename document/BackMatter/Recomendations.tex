\begin{recomendations}

\subsection*{Optimización y Escalabilidad}
\begin{itemize}
    \item Explorar métodos avanzados para optimizar la representación y comparación de Árboles de Sintaxis Abstracta (AST), buscando reducir el tiempo de procesamiento y los requerimientos computacionales en grandes conjuntos de datos. Esto podría incluir:
    \begin{itemize}
        \item \textbf{Paralelización de Procesos:} Implementar algoritmos paralelos o distribuidos para dividir el análisis en múltiples núcleos de procesamiento o máquinas.
        \item \textbf{Compresión de Estructuras:} Diseñar representaciones más compactas de los AST, como árboles binarios optimizados, que permitan una comparación más rápida y eficiente.
        \item \textbf{Preprocesamiento Inteligente:} Incorporar mecanismos de filtrado previo que identifiquen rápidamente fragmentos irrelevantes o claramente distintos, reduciendo el tamaño efectivo del conjunto de datos.
    \end{itemize}
\end{itemize}

\subsection*{Ampliación de la Herramienta}
\begin{itemize}
    \item Expandir la compatibilidad de la herramienta para incluir múltiples lenguajes de programación. Esto requeriría:
    \begin{itemize}
        \item Adaptación de gramáticas específicas para cada lenguaje soportado.
        \item Desarrollo de un módulo de detección automática del lenguaje de entrada.
        \item Evaluación de la efectividad de los análisis en lenguajes dinámicos (como Python) y lenguajes fuertemente tipados (como Java o C++).
    \end{itemize}
    \item Incorporar módulos adicionales para analizar similitudes en paradigmas específicos, como programación funcional o programación orientada a objetos.
    \item Diseñar una interfaz gráfica intuitiva que facilite la interacción con la herramienta, haciendo más accesibles sus funcionalidades para usuarios no técnicos.
\end{itemize}

\subsection*{Educación y Ética}

 Utilizar los hallazgos de esta investigación para fomentar prácticas éticas en la programación. Colaboración con instituciones educativas para integrar estas herramientas y contenidos en los planes de estudio de cursos de programación y ética profesional.

\subsection*{Monitoreo y Actualización de la Herramienta}
\begin{itemize}
    \item Establecer un ciclo continuo de evaluación y actualización para mantener la relevancia y precisión de la herramienta. Esto incluiría:
    \begin{itemize}
        \item Recolectar datos de uso reales para identificar patrones comunes de similitud y áreas de mejora.
        \item Mantener una base de datos de gramáticas de lenguajes actualizada, incluyendo soporte para nuevos lenguajes o versiones.
    \end{itemize}
\end{itemize}

\subsection*{Contribución Abierta}
\begin{itemize}
    \item Publicar los resultados y herramientas desarrolladas bajo licencias de código abierto, permitiendo a la comunidad académica e industrial contribuir a su mejora y expansión.
    \item Crear un repositorio público para compartir datasets, modelos y resultados de evaluación, fomentando investigaciones futuras en el campo.
\end{itemize}

\end{recomendations}

\begin{resumen}
 Esta investigación aborda el problema de la detección de similitudes en código fuente, específicamente en proyectos de C\#. El proceso comienza con la extracción del Árbol de Sintaxis Abstracta (AST) de diversos proyectos de primer año de programación, permitiendo una representación estructural del código. A partir de estos AST, se extraen características relevantes que describen los elementos del código. Estas características se modifican para crear variaciones y, posteriormente, se agrupan en pares para su análisis mediante técnicas de aprendizaje automático. Se realizaron pruebas con distintos algoritmos de clustering y redes neuronales. Finalmente, se optó por utilizar técnicas como One-Class SVM e Isolation Forest, las cuales demostraron ser más eficaces en la identificación de similitudes y anomalías en los proyectos de código. Este enfoque permite una evaluación más precisa y robusta en nuestro problema, proporcionando una herramienta útil para la detección de plagio.
\end{resumen}

\begin{abstract}
	This research addresses the issue of detecting similarities in source code, specifically in C# projects. The process begins with the extraction of the Abstract Syntax Tree (AST) from various first-year programming projects, allowing for a structural representation of the code. From these ASTs, relevant features are extracted that describe the elements of the code. These features are modified to create variations, which are then grouped in pairs for analysis using machine learning techniques. Tests were conducted with different clustering algorithms and neural networks. Ultimately, techniques such as One-Class SVM and Isolation Forest were chosen, which demonstrated greater effectiveness in identifying similarities and anomalies in the code projects. This approach allows for a more precise and robust evaluation in our problem, providing a useful tool for plagiarism detection.
\end{abstract}